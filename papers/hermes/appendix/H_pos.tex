\section{Details of Position Re-Indexing}

Inspired by StreamingVLM's strategy of managing positional stability in streaming scenarios~\cite{xu2025streamingvlmrealtimeunderstandinginfinite},  we adopt a unified left-compaction re-indexing scheme to eliminate positional gaps introduced by KV-cache pruning while preserving the semantic anchoring of the system prompt. Concretely, system text tokens are kept fixed to provide a stable textual anchor, whereas retained video tokens are re-indexed in a left-compact manner and placed contiguously after the static prefix. To reuse cached key states without re-computation, we further apply a delta-based rotary correction that compensates for the positional displacement.

\subsection{Re-indexing for LLaVA-OV (1D RoPE)}
\label{app:1d_pos}

LLaVA-OV employs standard 1D RoPE, where each token is associated with a scalar positional index $p$. Therefore, we perform left-compaction of the 1D indices: the system prefix positions remain unchanged, while the retained positions of video tokens are reassigned to form a dense contiguous segment immediately following the fixed prefix.

Let \texttt{offset} denote the length of the system prompt prefix tokens, and let
\[
\mathcal{P} = \{ p_0 < p_1 < \cdots < p_{N-1} \}
\]
be the sorted set of retained video token positions (excluding the fixed prefix). For a retained video token originally at position $p_{\mathrm{old}} \in \mathcal{P}$, its compacted 1D position is defined as
\begin{equation}
p_{\mathrm{new}}
=
\texttt{offset}
+
\operatorname{rank}_{\mathcal{P}}\!\left(p_{\mathrm{old}}\right).
\end{equation}
This mapping removes gaps while preserving the original temporal ordering along the stream, and ensures that the video region occupies a dense range directly after the static text region.

To align cached key states with the updated positions, we avoid re-generating keys and instead apply a rotary delta correction induced by the positional shift. For a cached key vector $\mathbf{k}_{\mathrm{old}}$ associated with position $p_{\mathrm{old}}$ and remapped to $p_{\mathrm{new}}$, we compute
\begin{equation}
\mathbf{k}_{\mathrm{new}}
=
\mathbf{k}_{\mathrm{old}}
\odot
\mathrm{RotaryDelta}\!\left(p_{\mathrm{old}}, p_{\mathrm{new}}\right),
\end{equation}
where the relative phase shift is
\begin{equation}
\mathrm{RotaryDelta}\!\left(p_{\mathrm{old}}, p_{\mathrm{new}}\right)
=
e^{i(p_{\mathrm{new}} - p_{\mathrm{old}})\boldsymbol{\theta}},
\end{equation}
and $\boldsymbol{\theta}$ denotes the RoPE frequency vector. This update preserves the correctness of attention under the new indexing while enabling direct reuse of the cached KV states.


\subsection{Re-indexing for Qwen2.5-VL (3D M-RoPE)}
\label{app:3d_pos}

For Qwen2.5-VL, video tokens are indexed by a 3D M-RoPE coordinate $\mathbf{p} = (p^{(t)}, p^{(h)}, p^{(w)})$, covering temporal and spatial dimensions. After pruning, the retained video tokens typically occupy sparse coordinates along each dimension $d \in \{t, h, w\}$. To eliminate the gaps without disturbing the monotonic ordering, we apply dimension-wise left-compaction independently along each axis, while keeping the system token prefix fixed.

Let
\[
\mathcal{P}^{(d)} = \{ p^{(d)}_0 < p^{(d)}_1 < \cdots < p^{(d)}_{N_{d}-1} \}
\]
denote the sorted set of retained coordinates along dimension $d$. For a token originally located at $p^{(d)}_{\mathrm{old}} \in \mathcal{P}^{(d)}$, its compacted coordinate is defined by its rank within $\mathcal{P}^{(d)}$, shifted by the fixed prefix \texttt{offset}:
\begin{equation}
p^{(d)}_{\mathrm{new}}
=
\texttt{offset}
+
\operatorname{rank}_{\mathcal{P}^{(d)}}\!\left(p^{(d)}_{\mathrm{old}}\right),
\qquad
d \in \{t, h, w\}.
\end{equation}
This procedure yields a dense and contiguous $(t,h,w)$ grid for the video tokens placed immediately after the static text region, thereby ensuring positional continuity while preserving the distinct semantic roles of temporal and spatial indices.

As in the 1D case, we reuse cached keys by applying a M-RoPE correction. Given a key $\mathbf{k}_{\mathrm{old}}$ associated with
\[
\mathbf{p}_{\mathrm{old}} = (p^{(t)}_{\mathrm{old}}, p^{(h)}_{\mathrm{old}}, p^{(w)}_{\mathrm{old}})
\]
and remapped to
\[
\mathbf{p}_{\mathrm{new}} = (p^{(t)}_{\mathrm{new}}, p^{(h)}_{\mathrm{new}}, p^{(w)}_{\mathrm{new}}),
\]
the corrected key is obtained as
\begin{equation}
\mathbf{k}_{\mathrm{new}}
=
\mathbf{k}_{\mathrm{old}}
\odot
\mathrm{RotaryDelta}\!\left(\mathbf{p}_{\mathrm{old}}, \mathbf{p}_{\mathrm{new}}\right),
\end{equation}
with the relative phase shift:
\begin{equation}
\mathrm{RotaryDelta}\!\left(\mathbf{p}_{\mathrm{old}}, \mathbf{p}_{\mathrm{new}}\right)
=
\operatorname*{Concat}_{d \in \{t, h, w\}}
\left(
e^{i
(
p^{(d)}_{\mathrm{new}} - p^{(d)}_{\mathrm{old}}
)
\boldsymbol{\theta}^{(d)}
}\right),
\end{equation}
where $\operatorname*{Concat}$ denotes the concatenation operation along the channel dimension, and $\boldsymbol{\theta}^{(d)}$ represents the rotary frequency vector corresponding to the channel section allocated for dimension $d$.
