
\section{StreamingVQA: Task Definition and Discussion} 
\label{sec:task}

This paper considers the problem of streaming video question-answering (\textbf{StreamingVQA}), where a model continuously processes a video stream and can respond to questions about past visual content at any moment. 
In this section, we formally define the task and outline the design principles for our proposed solution.

Given a video stream $\mathcal{V}^T := [v_1, v_2, ..., v_T]$ consisting of $T$ frames and a set of $N$ questions $\mathcal{Q} := \{q_1, q_2, \dots, q_N\}$,
StreamingVQA aims to answer a question $q_i$ at any time step $t~(1 \le t \le T)$, using only the frames seen up to that point, $\mathcal{V}^t := [v_1, v_2, ..., v_t]$.


\noindent\textbf{Discussion-I: StreamingVQA {\em vs.}~OfflineVQA.} StreamingVQA involves continuously analyzing an incoming video stream and answering questions based on the observed visual content at any moment. 
In contrast, conventional video question-answering models~\citep{frozen_bilm,video_chatgpt,llava_next_video,llava_onevision} operate in an offline mode, referred to as OfflineVQA.
The two paradigms differ in that: 
1) StreamingVQA processes a continuous video stream, while OfflineVQA handles a predefined video input, and 
2) StreamingVQA allows questions to be asked at any point during the stream, whereas OfflineVQA processes questions only after the entire video has been viewed.
Notably, OfflineVQA can be considered a special case of StreamingVQA, where all questions are posed after the video is fully processed. 


Conventional approaches typically employ a visual encoder~\citep{clip,siglip,fang2023eva} and a projection module~\citep{llava_next_video,blip2} to process video frames~($\mathcal{V}^t$).
The output is concatenated with the tokenized question to form a sequence $[\mathcal{V}_t, q_i]$~\footnote{We maintain the original notation for simplicity.}, which is then passed to an LLMs to predict an answer.
However, this approach is impractical due to the high computational cost associated with processing a large number of frames ($T$).


A common workaround is sparse frame sampling~\citep{video_chatgpt, llava_next_video, llava_onevision}, 
but this introduces new problems:
(i) loss of critical visual information, leading to incomplete or inaccurate responses, and
(ii) the need to reprocess frames for different questions, since questions asked at different time points require distinct frame samples. This becomes increasingly inefficient as $T$ and $N$ grow.

Given these challenges, current OfflineVQA methods fall short when applied to StreamingVQA scenarios. Therefore, designing a new approach optimized for StreamingVQA is crucial to handling video streams more efficiently, enabling real-time question answering and unlocking more interactive video analysis applications.


\noindent\textbf{Discussion-II: Design Principles for Efficient StreamingVQA.} 
To tackle the aforementioned challenges, we can exploit the causal nature of the LLM decoder to avoid redundant computations and strike a balance between accuracy and speed. 
During attention calculations, causal masking prevents the model from accessing future tokens, ensuring that video tokens are encoded independently of the questions. This allows us to \textit{decouple video encoding from question-answering}.

For video encoding, we leverage the KV-Cache optimization to accelerate inference.
However, as number of frames grows large, handling the massive number of video tokens becomes increasingly inefficient and may exceed the model’s capacity~\citep{lm_infinite,streamingllm}.
To address this, we adopt a sliding-window attention mechanism~\citep{lm_infinite}, which limits the attention scope to only the most recent frames.

Regarding question-answering, Video KV-Caches are stored and can be reused as context to answer different questions. 
However, long video sequences produce a substantial amount of KV-Caches, leading to excessive GPU memory consumption, computational overhead, and unnecessary distractions if all are used.
To address this, we introduce an efficient retrieval method that selects the most relevant video key-value vectors from the video KV-Caches. These selected vectors then serve as context, enabling efficient and scalable StreamingVQA.




\section{
\ourmethod: \underline{Re}trieve In-context Video \underline{KV}-Cache
}
\label{sec:method}

This section introduces \textbf{\ourmethod}, an approach that integrates seamlessly with a Video-LLM to enable efficient StreamingVQA without requiring additional training.
Overall, \ourmethod~efficiently encodes the video stream, maintains its KV-Caches, retrieves relevant caches based on the given question, and uses them for accurate question-answering.

\begin{figure}[t]
\centerline{\includegraphics[width=.85\linewidth]{figures/src/framework.pdf}}
\caption{
\textbf{Overview of~\ourmethod.}
We modify the attention mechanism in Decoder-based Video-LLMs:
(a) The video stream is encoded with sliding-window attention (Equation~\ref{equ:video_encoding}), with out-of-window Video KV-Caches offloaded to RAM or disk.
(b) Upon receiving a question, relevant key-value vectors are retrieved based on cosine similarity, with compressed vectors to accelerate retrieval (Equation~\ref{equ:retrieval}).
(c) The retrieved key-value vectors are reloaded onto the GPU and utilized for autoregressive answer generation (Equation~\ref{equ:answer_generation}).
}
\label{fig:framework} 
\end{figure}

We aim to enable Video-LLMs, trained on limited frames, 
to perform StreamingVQA \textbf{without additional training}.
As shown in Figure~\ref{fig:framework}, 
the proposed \ourmethod~has three components: 
video stream encoding, video KV-Cache retrieval, and question-answering using the retrieved key-value vectors.

\noindent\textbf{Video Stream Encoding with Sliding-window Attention.} 
We encode the video stream $\mathcal{V}^T$ incrementally, processing it chunk by chunk.
At each step, the inputs include past key-value vectors $\mathbf{P} = \{(\mathbf{k}_j, \mathbf{v}_j)\}_{j=1}^{l_P}$ and the current tokens $\mathbf{X}=\{\mathbf{t}_{i+l_P}\}_{i=1}^{l_X}$, 
where $l_P$ denotes the lengths of past key-values, and $l_X$ refers to the chunk size.
The local key-value vectors within a window $l_L$ can thus be derived as 
$\mathbf{L} = \mathbf{P}_{[l_P - l_L + 1 : l_P]}$.
The attention calculation is then formulated as:
\begin{equation}
    \mathbf{O} = \text{Attn}\left(\mathbf{W_Q}\mathbf{X}, [\mathbf{L}_k, \mathbf{W_K}\mathbf{X}], [\mathbf{L}_v, \mathbf{W_V}\mathbf{X}]\right),
    \label{equ:video_encoding}
\end{equation}
where $\mathbf{W_Q}$, $\mathbf{W_K}$, and $\mathbf{W_V}$ are the attention layer parameters, $\mathbf{L}_k$ and $\mathbf{L}_v$ correspond to the key and value vectors in $\mathbf{L}$. All video KV-Caches are stored for future retrieval. For extremely long videos, we manage memory constraints by offloading KV-Caches to RAM or disk, as in~\citep{infllm}.

\noindent\textbf{External Video KV-Cache Retrieval.} 
Here, we utilize an external CLIP-like model~\citep{clip, siglip} to retrieve question-relevant video KV-Cache, primarily as a baseline to assess whether retrieval can enhance VideoQA performance, as demonstrated in Section~\ref{sec:experiments}. 
Specifically, a CLIP-like model transformers each video frame into a vector $\mathbf{v} = f_v(v) \in \mathrm{R}^D$, where $f_v$ represents the visual encoder, $D$ denotes the vector dimension. Similarly, the question is encoded as $\mathbf{q} = f_t(q) \in \mathrm{R}^D$, where $f_t$ is the text encoder. We then compute the cosine similarity between the embeddings of frame and question: 
\begin{equation}
    \text{Sim}(\mathbf{v}, \mathbf{q}) = \frac{\mathbf{v} \cdot \mathbf{q}}{\tau~||\mathbf{v}||~||\mathbf{q}||}
    \label{equ:retrieval}
\end{equation} 
where $\tau$ is a learnable temperature parameter. This similarity is calculated at the frame level, rather than at the token level. Alternatively, we can group $b$ consecutive frames into blocks by averaging their frame vectors and then compute block-level similarity scores. Finally, the $r$ most relevant video frames or $\lceil r/b \rceil$ video blocks are retrieved. The corresponding video KV-Cache, denoted as $\mathbf{R}$, is subsequently loaded onto the GPU for question-answering.

\noindent\textbf{Internal Video KV-Cache Retrieval.} Building on recent advancements in handling long sequences with LLMs~\citep{infllm, quickllama, em_llm}, we further explore using self-attention layers within Video-LLMs for retrieval. Similar to external retrieval, internal retrieval is still performed at the level of video frames or blocks.

During video modeling, the average of the key vectors of a frame is computed as its representative frame vector: $\mathbf{v} = \frac{1}{N_f} \sum_{j=1}^{N_f} \mathbf{k}_j \in \mathrm{R}^{D'}$, where $N_f$ is the number of tokens per frame, and $\mathbf{k}_j$ is the $j$-th key vector. 
To reduce computational overhead, we do not differentiate between attention heads and instead concatenate them into a single vector, with $D'$ as the resultant dimension. Similarly, the question vector is computed as $\mathbf{q} = \frac{1}{N_q} \sum_{k=1}^{N_q} \mathbf{q}_{k} \in \mathrm{R}^{D'}$, where $N_q$ is the number of tokens in the question, 
and $\mathbf{q}_{k}$ is its $k$-th query vector. The similarity computation and video KV-Cache retrieval are identical to that of external retrieval, except that $\tau$ is set to 1.

Note that, internal retrieval offers several advantages over external retrieval. 
First, it operates independently within each self-attention layer, allowing different layers to retrieve different video blocks.\footnote{For simplicity, we omit the layer index in the above explanation.}
This allows for a broader capture of video context. 
Additionally, internal retrieval reuses already computed hidden representations and does not introduce extra parameters, which reduces the computational overhead compared to external retrieval.

\noindent\textbf{Question-Answering Using Retrieved KV.} The retrieved Video KV-Caches serve as the context for video question-answering. Formally, the attention calculation is formulated as:
\begin{equation}
    \mathbf{O} = \text{Attn}\left(\mathbf{W_Q}\mathbf{X}, [\mathbf{R}_k, \mathbf{W_K}\mathbf{X}], [\mathbf{R}_v, \mathbf{W_V}\mathbf{X}]\right),
    \label{equ:answer_generation}
\end{equation} 
where $\mathbf{X}$ represents either the question tokens or the current token being decoded, and $\mathbf{R}_k$ and $\mathbf{R}_v$ are the key and value vectors from the context, which includes the retrieved video, question, and previously generated tokens. 

\noindent\textbf{Positional Encoding.}
Our baseline Video-LLMs employ Rotary Position Embeddings (RoPE)~\citep{su2024roformer}, a commonly used relative positional encoding method.
Our video streaming encoding process follows LM-Infinite~\citep{lm_infinite}, where RoPE operates normally within the local window but is constrained by a ``distance ceiling'' for more distant tokens. 
For question-answering, we do not account for the original positions of the retrieved KV-Caches, as handling unseen distances among tokens presents significant challenges~\citep{lm_infinite}.
Instead, we treat these retrieved tokens as regular consecutive tokens. 
We also experimented with a static variation from Inf-LLM~\citep{infllm}, where all retrieved tokens are assigned the same position. 
Our results show that applying standard RoPE to retrieved video tokens leads to better performance, likely due to the importance of capturing temporal information in video comprehension.
